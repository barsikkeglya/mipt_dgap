\documentclass[a4paper,12pt]{article}
\usepackage[T2A]{fontenc}
\usepackage[utf8]{inputenc}
\usepackage[english,russian]{babel}
\usepackage{circuitikz}
\usepackage{wrapfig}
\usepackage{makecell}
\usepackage{tabularx}
\usepackage{graphicx}
\usepackage{gensymb}
\usepackage{cancel} %cancel symbol
\usepackage{amsmath,amsfonts,amssymb,amsthm,mathtools}

\setlength{\parskip}{\baselineskip}%
\setlength{\parindent}{0pt}%

%tikz (draw)

\usepackage{tikz}

%tikz libraries

\usetikzlibrary{intersections}
\usetikzlibrary{arrows.meta}
\usetikzlibrary{calc,angles,positioning}

\usepackage{float}

\parindent=0ex

%\graphicspath{ {C:/Users/George/Documents/MIPT_TEX/} }

\newcommand{\R}{{\mathbb R}}
\newcommand{\N}{{\mathbb N}}
\newcommand{\fancy}[1]{{\mathbb{#1}}}
\DeclareMathOperator{\sgn}{sgn}
\renewcommand\thesection{\arabic{section}}
\newcommand{\uni}{\cup}
\newcommand{\inter}{\cap}
\newtheorem{problem}[]{Задача}
\theoremstyle{definition}
\newtheorem*{solution}{Решение}



\begin{document}
    %\tableofcontents

    \pagenumbering{arabic}
    \subsection*{Холодильная машина}
    
    \begin{align}
        \eta_\text{хол} =\frac{Q_{\text{хол}}}{A} =\frac{|Q_\text{хол}|}{Q_\text{наг}-|Q_\text{хол}|} \\
        \eta_\text{тепл} = \frac{Q_{\text{нагр}}}{A} \\
        \eta_\text{хол} = \eta_\text{тепл}-1
    \end{align}
    \subsection*{Задачи}
        \begin{problem}[3.25]
            \begin{figure}[H]
                \centering
                \includegraphics*[width=1\textwidth]{img/3.25.png}
            \end{figure}
        \end{problem}
        \begin{solution}
            \begin{eqnarray*}
                T_\text{н}:=T_\text{в} \qquad T_\text{х}:=T_\text{л}
            \end{eqnarray*}
            Температура холодильника $T_\text{х}$ меняется.
            \begin{align*}
                \frac{Q_\text{н}}{Q_\text{х}}=&\frac{T_\text{н}}{T_\text{х}}\\
                Q_\text{н}=cm_1dT, \qquad &Q_\text{х}=qdm \\ 
                \frac{dT}{T}=&\frac{qdm}{m_1cT_\text{х}} \\ 
                \ln \frac{T_1}{T}=&\frac{-m_2q}{m_1cT_2} \\
                \Aboxed{ T=T_1\exp&{\frac{-m_2q}{m_1cT_2}} } \\
                \Aboxed{A=Q_\text{н}-|Q_\text{хол}|=&cm_1|T-T_1|-qm_2}
            \end{align*}
        \end{solution}
        
        \begin{problem}[T2]
            \begin{figure}
                \centering
                \includegraphics*[width=1\linewidth]{img/t2.png}
            \end{figure}
        \end{problem}
        \begin{solution}
            Индекс <<h>> обозначает нагреватель, <<c>> -- холодильник. Заметим, что процесс изохорический
            \begin{align*}
                \frac{dQ_h}{dQ_c}=\frac{dU_h}{dU_c}&\\ 
                \frac{T_h}{T_c}=\frac{dT_h}{-dT_c} \rightarrow \int_{T_0}^{T_2}\frac{dT_h}{T_h}=-\int_{T_0}^{T_1}\frac{dT_c}{T_c}&\\
                \ln \frac{T_2}{T_0}=\ln\frac{T_0}{T_1}\rightarrow T_2=\Aboxed{\frac{T_0^2}{T_1}}=450 \text{K}&\\
                A=Q_h-|Q_c|=\Delta U_h-|\Delta U_c|&\\
                \Delta U_h=C_V |T_2-T_0| \quad \Delta U_c=C_V|T_1-T_0|&\\
                A=\Aboxed{\frac{5}{2}R(T_2-T_0)-\frac{5}{2}R(T_0-T_1)}=1038 \text{J}
            \end{align*}
        \end{solution}

        \begin{problem}[3.43]
            \begin{figure}[H]
                \centering
                \includegraphics*[width=1\linewidth]{img/3.43.png}
            \end{figure}
        \end{problem}
        \begin{solution}
            \begin{align*}
                P_{transfer}=\alpha(t_2-t_1), \quad P_{transfer}=P\\ 
                \dot{Q_h}-|\dot{Q_c}|=\dot{A}=\eta\alpha(t_2-t_1)\\
                \frac{\dot{Q_h}}{\dot{Q_c}}=\frac{Q_h}{Q_c}=\frac{t_x}{t_2}\\
                \dot{Q_h}=\alpha(t_x-t_2)
            \end{align*}
        \end{solution}
\end{document}