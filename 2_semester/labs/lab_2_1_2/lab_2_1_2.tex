\documentclass[a4paper,12pt]{report}
\usepackage[T2A]{fontenc}
\usepackage[utf8]{inputenc}
\usepackage[english,russian]{babel}
\usepackage{circuitikz}
\usepackage{wrapfig}
\usepackage{makecell}
\usepackage{tabularx}
\usepackage{graphicx}
\usepackage{gensymb}
\usepackage{cancel} %cancel symbol
\usepackage{amsmath,amsfonts,amssymb,amsthm,mathtools}
\usepackage{pgfplots}
\usepackage{pgfkeys}
%\usepackage[margin=3cm]{geometry}
\pgfplotsset{compat=1.12}
\usepackage{mathrsfs}

\usepackage[pagestyles]{titlesec}
\titleformat{\chapter}[display]{\normalfont\huge\bfseries\centering}   {\chaptertitlename\ \thechapter}{20pt}{\Huge}
\titlespacing{\chapter}{0pt}{-48pt}{1cm}

%tikz (draw)

\usepackage{tikz}
\usepackage{pstricks-add}
%tikz libraries

\usetikzlibrary{intersections}
\usetikzlibrary{arrows.meta}
\usetikzlibrary{calc,angles,positioning}
\usetikzlibrary{arrows}
\usepackage{float}

\parindent=0ex
\setlength{\parskip}{\baselineskip}%
\setlength{\parindent}{0pt}%

\graphicspath{ {C:/Users/George/Documents/MIPT_TEX/} }

\newcommand{\R}{{\mathbb R}}
\newcommand{\N}{{\mathbb N}}
\newcommand{\fancy}[1]{{\mathbb{#1}}}
\DeclareMathOperator{\sgn}{sgn}
\newtheorem{problem}{Задача}[]
\newenvironment{sol}{\paragraph{Решение}}{}
\renewcommand\thesection{\arabic{section}}
\newcommand{\uni}{\cup}
\newcommand{\inter}{\cap}

\begin{document}
	\begin{titlepage}
	\begin{center}
		МОСКОВСКИЙ ФИЗИКО-ТЕХНИЧЕСКИЙ ИНСТИТУТ (НАЦИОНАЛЬНЫЙ ИССЛЕДОВАТЕЛЬСКИЙ УНИВЕРСИТЕТ) \\
		
		
		\hfill \break
		Факультет обшей и прикладной физики\\
		\vspace{2.5cm}
		\LARGE{\textbf{Отчет о выполнении второй лабораторной работе по информатике \\Асимптотическая сложность алогритмов сортировки}}\\
		\hfill \break
		\\
	\end{center}
	
	\begin{flushright}
		Выполнил:\\
		Студент гр. Б02-304\\
		Головинов. Г.А.
	\end{flushright}
	
	\vfill
	
	\begin{center}
		\includegraphics[width=0.15\linewidth]{uni}
	\end{center}
	
	\begin{center} Долгопрудный, 2024 \end{center}
	
	\thispagestyle{empty}
	
\end{titlepage}
	\newpage
	\pagenumbering{arabic}
    
    \subsection*{Аннотация}
        \paragraph*{Цель работы:} определение отношения $C_p/C_V$ для воздуха или углекислого газа по измерению давления в стеклянном сосуде. Измерения производятся сначала для адиабатического расширения гааз, а затем после нагревания сосуда и газа до комнатной температуры.
        \paragraph*{В работе используются:} стеклянный сосуд; U-образный жидкостный манометр; резиновая группа; газгольдер с углекислым газом.
    \section{Основные теоретические сведения}
    \subsection*{Экспериментальная установка}
        \begin{figure}[H]
            \includegraphics*[width=0.75\textwidth]{ustanovka.png}
            \centering
            \caption{Установка для определения отношения $C_p/C_V$}
        \end{figure}
        На рисунке сосуд А (объем $\approx$ 20 л.), кран К, U-образный жидкостный манометр М. Кран К$_1$ и резиновая груша позволяют создавать избыточное давление воздуха. Углекислый газ подается из газгольдера.

        В начале опыта газ в сосуде А находится при комнатной температуре $T_1$, давлении $P_1$, несколько превышающем атмосферное давление $P_0$. После открытия крана К давление и температура газа будут понижаться.

        Этот процесс приближенно можно считать адиабатическим. Приближение основано на том, что равновесие в газах по давлению наступает намного быстрее, чем равновесие по температуре. Соответственно будем считать $\Delta t_P$ -- время установления равновесия по давлению сильно меньше, чем $\Delta t_T$ -- время установления равновесия по температуре.

        Необходимо также учесть тот факт, что на это предположение влияет размер клапана, если он слишком мал, то предположение неверно. Поэтому если $\Delta t_P \ll \Delta t_T$, то любой процесс за время $\Delta t$ между интервалами установления можно считать приближенно адиабатическим.

    \subsection*{Уравнение адиабаты}
        Первое начало термодинамики:
        \begin{equation}
            \label{1st law thermodynamics}
            \delta Q + \delta A^{\text{над газом}}=dU
        \end{equation}
        При адиабатическом процессе $\delta Q = 0$, Тогда 
        \begin{equation}
            \label{1st law enhanced}
            \delta Q = dU+\delta A = 0
        \end{equation}
        где $\delta A$ -- работа газа. В свою очередь изменение внутренней энергии и работа идеального газа выражаются
        \begin{equation}
            dU=C_VdT
            \label{dU}
        \end{equation}
        \begin{equation}
            \delta A=pdV
            \label{delta A}
        \end{equation}
        Далее нам потребуется уравнение Менделеева-Клапейрона:
        \begin{equation}
            pV=\nu RT
            \label{pV}
        \end{equation}
        Для удобства будущих рассчетов будем использовать $\nu = 1$, подставим уравнение Менделеева-Клапейрона и \eqref{dU}, \eqref{delta A} в уравнение \eqref{1st law enhanced} и получим
        \begin{equation}
            C_V\frac{dT}{T}+R\frac{dV}{V}=0
            \label{before integration}
        \end{equation}
        при постоянной $C_V$ уравнение \eqref{before integration} можно проинтегрировать:
        \begin{eqnarray}
            C_V\ln T+R\ln V=const\\ 
            TV^{R/C_V}=const
        \end{eqnarray}
        Используя еще раз соотношение \eqref{pV}, а также уравнение Майера:
        \begin{equation}
            \label{Mayer}
            C_p-C_V=R
        \end{equation}
        получим
        \begin{equation}
            \label{final gamma}
            pV^\gamma=const
        \end{equation}
        где $\gamma=C_p/C_V$ -- называется \emph{показателем} адиабаты.

        Нам в работе удобно перейти к переменным $p$ и $T$:
        \begin{equation}
            \label{pT-final}
            \left( \frac{p_1}{p_2} \right)^{\gamma-1}=\left( \frac{T_1}{T_2} \right)^\gamma
        \end{equation}
        Здесь мы обозначаем индексом <<1>> состояние до открытия крана, а <<2>> -- состояние после открытия крана и установления равновесия давлений. 
        
        После адиабатического (с учетом приближения) расширения газа $p_2=p_0$ -- атмосферное давление. Температура $T_2$ будет ниже комнатной, так как работа осуществляется за счет внутренней энергии газа. Когда мы закроем кран, газ начнет медленно (относительно изменения давления при расширении) нагреваться изохорически до комнатной температуры.
        
        При изохорическом нагревании выполняется соотношение
        \begin{equation}
            \frac{p_2}{T_2}=\frac{p_3}{T_1}   
        \end{equation}
        здесь $p_3$ -- давление после нагревания, $T_1$ -- комнатная температура (такая же, что и перед открытием клапана). С помощью этого соотношения можно выбросить из уравнения \eqref{pT-final} отношение температур и получить:
        \begin{equation}
            \label{P-final}
            \left(\frac{p_3}{p_2}\right)^\gamma = \left(\frac{p_1}{p_2}\right)^{\gamma-1}
        \end{equation}
        Учитывая, что давление в состоянии <<2>> равно атмосферному ($p_0$), то показатель адиабаты можно найти используя соотношение:
        \begin{equation}
            \label{gamma}
            \gamma = \frac{\ln(p_1/p_0)}{\ln(p_1/p_3)}
        \end{equation}
        Итого, показатель адиабаты находится с помощью 3-х неизвестных: атмосферного давления и давлений в состояниях <<1>> и <<3>>. Их можно можно выразить как атмосферное давление плюс некоторая небольшая разница, которую мы будем измерять с помощью U-образного манометра М.
        \begin{equation*}
            p_1=p_0+\rho gh_1, \qquad p_3=p_0+\rho gh_3
        \end{equation*}
        где $\rho$ -- плотность жидкости, $h_1$ -- высота столба в состоянии <<1>>, $h_3$ -- в состоянии <<3>>. Далее преобразуем уравнение $\gamma$ с учетом этих соотношений:
        \begin{equation*}
            \gamma=\frac{\ln([p_0+\rho gh_1]/p_0)}{\ln([p_0+\rho gh_1]/[p_0+\rho gh_3])}=\frac{\ln(1+\rho gh_1/p_0)}{\ln (1+\rho gh_1/p_0)-\ln(1+\rho gh_3/p_0)}\\ 
        \end{equation*}
        Считая, что разница между атмосферным давлением и давлением в системе сильно меньше чем само атмосферное давление, логарифмы можно разложить в ряд Тейлора. Примем следующие обозначения:
        \begin{equation*}
            \frac{\rho gh_1}{p_0}=x_1\rightarrow0, \qquad \frac{\rho g h_3}{p_0}=x_3\rightarrow 0
        \end{equation*}
        \begin{align}
            \nonumber \gamma&=\frac{\ln(1+x_1)}{\ln(1+x_1)-\ln(1+x_3)}=\\
            \nonumber&=\frac{x_1-\frac{1}{2}x_1^2+o(x_1^2)}{x_1-\frac{1}{2}x_1^2+o(x_1^2)-(x_3-\frac{1}{2}x_3^2+o(x_3^2))}\approx\\ 
            &\approx \frac{x_1}{x_1-x_3}=\frac{h_1}{h_1-h_3} \label{final gamma via h}
        \end{align}
    \subsection*{Время вытекания газа}
    Вязкостью газа пренебрежем.

    При открытии клапана К по газу со скоростью звука будет распространятся волна, которая дойдет до дна за $L/c$, где $L$ -- высота сосуда, $c$ -- скорость звука. Через несколько таких интервалов времени можно считать, что весь газ придет в движение и будет двигаться с некоторой скорость $v$. Этот процесс будем считать квазистационарным. Скорость $v$ можно найти из уравнения Бернулли для несжимаемой среды (изменением плотности пренебрегаем в силу небольшой разницы давлений газа и атмосферы).
    \begin{equation}
        \label{v}
        v=\sqrt{\frac{2(p-p_0)}{\rho_0}}
    \end{equation}
    За время $dt$ из сосуда через отверстие площадью $S$ вытечет масса $dm=\rho_0 S vdt$, где плотность взята при атмосферном давлении.

    В сосуде объема $V_0$ давление за это же время снизится на $dp$, масса газа уменьшится на величину
    \begin{equation*}
        dm=V_0d\rho=\frac{V_0}{c^2}dp
    \end{equation*}
    здесь было использовано определение адиабатической скорости звука:
    \begin{equation*}
        c^2=\left(\frac{\partial p}{\partial \rho}\right)_S
    \end{equation*}
    Составим баланс вытекающей массы и остающейся в сосуде получим:
    \begin{equation*}
        \frac{dp}{\sqrt{p-p_0}}=-\frac{\sqrt{2\rho}Sc^2}{V_0}dt
    \end{equation*}
    Интегрируя получим:
    \begin{equation}
        \label{t_p}
        t_p=\frac{V_0}{Sc}\sqrt{\frac{2(p-p_0)}{\gamma p_0}}
    \end{equation}
    Используя приблизительные данные установки получим $t_p\approx 0,1$ с.

    Важно заметить, что из-за акустических колебаний при малых интервалах открытия клапана все-равно возникает большой разброс при измерениях. 

    \subsection*{Нагревание газа от стенок сосуда} Теперь необходимо оценить скорость теплообмена между газом и стенками сосуда.

    Будем считать время выравнивания давления порядка 1 с. В течение такого времени глубина прогревания невелика, значительно меньше размеров сосуда, поэтому нагревание приближенно можно считать одномерным. Процесс изменения температуры в пространстве описывается уравнением теплопроводности (в частных приводных), точное решение которого сложно даже для одномерного случая, поэтому ограничимся оценкой.

    Будем использовать коэффициент температуропроводности
    \begin{equation*}
        \chi=\frac{\kappa}{\rho c_p}
    \end{equation*}
    где $\kappa$ -- коэффициент теплопроводности, $c_p$ -- теплоемкость газа при постоянном давлении (на единицу массы), $\rho$ -- его плотность. Размерность $\chi$ в СИ: м$^2$/c.

    Решение -- есть функция от безразмерного параметра $x^2/\chi t$, где $x$ -- координата, $t$ -- время. Одному и тому же значению параметра будет соответствовать одна и та же температура. При $x^2/\chi t=1 \Leftrightarrow x=\sqrt{\chi t}$ в точном решении задачи о нагревании полупространства температура равна примерно среднему значению между постоянной температурой стенки и среды. Это значение $x$ можно использовать для оценки толщины слоя газа, нагревшегося от стенки.
    
    Время $t$ возьмем равным 0,5 с. Для воздуха $c_p=0,99$ Дж/(г$\cdot$К), $\rho$=1,29 кг/м$^3$, $\kappa=2,50\cdot10^{-2}$ Вт/(м$\cdot$К). По этим данным $\chi=0,19$ см$^2$/с, при $t=0,5$ с. глубина слоя составит $x=0,3$ см. Учитывая радиус сосуда $12,5$ см доля нагретого воздуха составит около 5\% от массы в сосуде. Это достаточно много, так как разница между 1- и 2-х атомными газами составляет примерно столько же.

    \subsection*{Охлаждение стеклянных стенок}
    Глубину охлаждения стенок сосуда можно оценить аналогично. Получится $x=0,045$ см. Учитывая намного более высокую теплоемкость стекла и малую толщину охлажденного слоя, изменение температуры стенок будет более чем в сто раз меньше, чем изменение температуры газа. Этим можно пренебречь (как вторым порядком малости). На предыдущую оценку это явление не повлияло.

    Бороться с достаточно большим отклонением из-за прогрева воздуха будем с помощью многократного повторения опыта при разных временах открытия клапана.

    \section*{Ход работы}
    Для начала необходимо убедиться, что внутри сосуда находится углекислый газ. Для этого мы провели 10-15 <<тестовых>> попыток эксперимента: наполняли сосуд дополнительным газом, не дожидаясь установления равновесия, открывали клапан. Таким образом часть углекислого газа в смеси можно повысить.

    Было проведено 10 опытов, в каждом из которых была получена зависимость $h(t)$ и значение показателя адиабаты $\gamma$.

    \section*{Обработка полученных результатов}

    Сначала рассмотрим результаты измерения зависимости $h(t)$. В процессе выполнения стало понятно, что время установления равновесия температуры находится в районе 2-3 минут.

    \begin{figure}[H]
        \centering
        \includegraphics*[width=1\linewidth]{h(t).png}
        \caption{Результаты измерений зависимости $h(t)$}
    \end{figure}    

    Некоторые эксперименты не совсем сходятся с основной группой. Например <<1>> и <<2>> (особенно <<1>>) оказались дальше всего. Это можно объяснить тем, что в сосуде еще оставалась достаточно большая часть воздуха, а также тем, что измерение зависимости $h(t)$ достаточно сложно, поэтому первые опыты оказались не самыми лучшими.
    
    Хорошо видно, что зависимости (особенно последние 5) имеют некоторую асимптоту. Можно считать, что после около 180-240с. температура в системе установилась до комнатной, так как последующее изменение давления много меньше малой разницы относительно $p_0$.

    Далее, после открытия крана на примерно 1с. и установления равновесия были получены значения для показателя адиабаты $\gamma$.

    \begin{figure}
        
        
    \end{figure}

    Систематическую погрешность измерения показателя адиабаты будем рассчитывать по формуле:

    \begin{equation}
        \label{sigma gamma}
        \sigma_\gamma=\gamma\sqrt{\left(\varepsilon_{0.05}\right)^2+\left(\frac{\sigma_{h1}}{h_1}\right)^2+\left(\frac{\sigma_{{h3}}}{h_3}\right)^2}
    \end{equation}

    Где $\varepsilon_{0.05}$ -- относительная погрешность 5\%, которая возникла в результате прогрева воздуха, $\sigma_{\Delta h}$ -- систематическая погрешность измерения высоты столба жидкости. Будем брать $\sigma_{h}=$0.1 cm

    \begin{table}[H]
        \centering
        \begin{tabular}{|c|c|c|c|}
            \hline
            Опыт & $h_1$, cm & $h_3$, cm & $\gamma$ \\
            \hline
            1 & 9.9 & 1.8 & 1.222 \\
            \hline
            2 & 9.1 & 1.8 & 1.246 \\
            \hline
            3 & 9 & 1.85 & 1.258 \\
            \hline
            4 & 9.3 & 1.9 & 1.257 \\
            \hline
            5 & 9.8 & 2.5 & 1.342 \\
            \hline
            6 & 9.8 & 2.3 & 1.307 \\
            \hline
            7 & 9.5 & 2.2 & 1.301 \\
            \hline
            8 & 9.4 & 2.1 & 1.288 \\
            \hline
            9 & 9.6 & 2.2 & 1.297 \\
            \hline
            10 & 9.5 & 2.2 & 1.301 \\
            \hline
        \end{tabular}
        \caption{Результаты измерения показателя адиабаты $\gamma$}
    \end{table}

    Табличный показатель адиабаты углекислого газа при температуре около 20C\degree \ равен 1.3. 
    
    Заметна также разница в результатах между первыми и последними опытами. В ходе работы с каждым опытом мы оставляли клапан открытым на все более малые промежутки времени, что позволяло увеличить точность измерений.

    Случайная погрешность измерения:
    \begin{equation*}
        \sigma_\gamma^{rnd}\approx0.034
    \end{equation*}

    Систематическая погрешность согласно формуле \eqref{sigma gamma}:

    \begin{equation*}
        \sigma_\gamma^{sys}\approx0.083
    \end{equation*}

    Итого результат измерения показателя адиабаты:

    \begin{align}
        \Aboxed{\gamma =& 1.282 \pm 0.089}
    \end{align}

    \paragraph*{Экстраполяция результатов при $\Delta t \rightarrow 0$}
    К сожалению достаточно сложно измерить время, которое был открыт клапан, поэтому будем использовать данные с большой погрешностью по времени. 

    \begin{figure}[H]
        \centering
        \begin{tikzpicture}[scale=1.3]
            \begin{axis}[
                ymajorgrids=true,
                xmajorgrids=true,
                xlabel={$\Delta t, \ s$},
                ylabel={$\gamma$},
                legend pos = north west,
                legend style={nodes={scale=1, transform shape}}, 
                legend image post style={mark=*}
            ]
            \addplot[
                only marks,mark=*,color=black,mark size = 1pt
            ]
            plot [error bars/.cd, y dir = both, x dir = both, y explicit, x explicit]
            table[meta=label, x=t, y=gamma, x error = xe, y error=ye]{
                gamma	t	xe	ye	label
                1.222222222	1	0.15	0.089	a
                1.246575342	0.7	0.15	0.089	a
                1.258741259	0.7	0.15	0.089	a
                1.256756757	0.7	0.15	0.089	a
                1.382352941	0.2	0.15	0.089	a
                1.342465753	0.4	0.15	0.089	a
                1.306666667	0.5	0.15	0.089	a
                1.301369863	0.5	0.15	0.089	a
                1.287671233	0.7	0.15	0.089	a
                1.297297297	0.5	0.15	0.089	a
                1.301369863	0.5	0.15	0.089	a
            };
            \end{axis}
        \end{tikzpicture}
        \caption{Экстраполяция $\Delta t \rightarrow 0$}
    \end{figure}
    Видно, что при более маленьком времени $\Delta t$ наблюдается более высокий показатель адиабаты $\gamma$. При апроксимации МНК получаем прямую $\gamma=k\Delta t + b$, где $k\approx-0.21$, $b\approx 1.41$. Поэтому при $\Delta t \rightarrow 0 \ \gamma\rightarrow 1.41$. Этот результат выше табличного $\gamma=1.30$, так как за малое время $\Delta t \rightarrow 0$ давление не успевает сравняться с атмосферным. 
    \section*{Обсуждение полученных результатов и выводы}
    
    В результате работы мы успешно измерили показатель адиабаты для углекислого газа при комнатной температуре. Результат получен с погрешностью $\approx 6.9\%$, что достаточно много. Однако больший вклад в погрешность внес нагрев воздуха (5\%, согласно расчетам в теоретической части).

    Погрешность оценена корректно, отклонение от табличных значений составила немного более 1.5\%. Возможно, оценка прогрева воздуха завышена и вклад в погрешность на самом деле ниже.

    В ходе работы возникли трудности оценки времени открытия клапана. Это связано в том числе и с тем, что воздух начинает проходить даже когда клапан еще не находится в открытом положении. Это усложняет точное определение времени открытия клапана, однако мы пытались его уменьшать с каждым опытом. Это сказалось на точности результатов: последние несколько опытов мы стабильно получали значения, близкие к табличным.

\end{document}