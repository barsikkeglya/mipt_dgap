\documentclass[a4paper,12pt]{report}
\usepackage[T2A]{fontenc}
\usepackage[utf8]{inputenc}
\usepackage[english,russian]{babel}
\usepackage{circuitikz}
\usepackage{wrapfig}
\usepackage{makecell}
\usepackage{tabularx}
\usepackage{graphicx}
\usepackage{gensymb}
\usepackage{cancel} %cancel symbol
\usepackage{amsmath,amsfonts,amssymb,amsthm,mathtools}
\usepackage{pgfplots}
\usepackage{pgfkeys}
%\usepackage[margin=3cm]{geometry}
\pgfplotsset{compat=1.12}
\usepackage{mathrsfs}

\usepackage[pagestyles]{titlesec}
\titleformat{\chapter}[display]{\normalfont\huge\bfseries\centering}   {\chaptertitlename\ \thechapter}{20pt}{\Huge}
\titlespacing{\chapter}{0pt}{-48pt}{1cm}

%tikz (draw)

\usepackage{tikz}
\usepackage{pstricks-add}
%tikz libraries

\usetikzlibrary{intersections}
\usetikzlibrary{arrows.meta}
\usetikzlibrary{calc,angles,positioning}
\usetikzlibrary{arrows}
\usepackage{float}

\parindent=0ex
\setlength{\parskip}{\baselineskip}%
\setlength{\parindent}{0pt}%

\graphicspath{ {C:/Users/George/Documents/MIPT_TEX/} }

\newcommand{\R}{{\mathbb R}}
\newcommand{\N}{{\mathbb N}}
\newcommand{\fancy}[1]{{\mathbb{#1}}}
\DeclareMathOperator{\sgn}{sgn}
\newtheorem{problem}{Задача}[]
\newenvironment{sol}{\paragraph{Решение}}{}
\renewcommand\thesection{\arabic{section}}
\newcommand{\uni}{\cup}
\newcommand{\inter}{\cap}

%\newcommand{\celsius}{$\ ^\circ C$}

\begin{document}
	\begin{titlepage}
	\begin{center}
		МОСКОВСКИЙ ФИЗИКО-ТЕХНИЧЕСКИЙ ИНСТИТУТ (НАЦИОНАЛЬНЫЙ ИССЛЕДОВАТЕЛЬСКИЙ УНИВЕРСИТЕТ) \\
		
		
		\hfill \break
		Факультет обшей и прикладной физики\\
		\vspace{2.5cm}
		\LARGE{\textbf{Отчет о выполнении второй лабораторной работе по информатике \\Асимптотическая сложность алогритмов сортировки}}\\
		\hfill \break
		\\
	\end{center}
	
	\begin{flushright}
		Выполнил:\\
		Студент гр. Б02-304\\
		Головинов. Г.А.
	\end{flushright}
	
	\vfill
	
	\begin{center}
		\includegraphics[width=0.15\linewidth]{uni}
	\end{center}
	
	\begin{center} Долгопрудный, 2024 \end{center}
	
	\thispagestyle{empty}
	
\end{titlepage}
	\newpage
	\pagenumbering{arabic}
    
    \section*{Аннотация}
        \paragraph*{Цель работы:} 1) определить изменения температуры углекислого газа при протекании через малопроницаемую перегородку при разных начальных значениях давления и температуры; 2) вычислить по результатам опытов коэффициенты $a$ и $b$ модели Ван-дер-Ваальса.
        \paragraph*{В работе используются:} трубка с пористой перегородкой; труба Дьюара; термостат жидкостный; дифференциальная термопара; мультиметр; балластный баллон; манометр.

    \section*{Основные теоретические сведения}
        Эффектом Джоуля---Томсона называется изменение температуры газа, просачивающегося из области высокого в область низкого давления в условиях тепловой изоляции. В разреженных газах (практически идеальных) при таком течении температура не меняется.

        В работе газ из области повышенного давления $P_1$ проходит через множество узких и длинных каналов пористой перегородки в область с атмосферным давлением $P_2$. Перепад давления $|\Delta P|=P_1-P_2$ из-за большого сопротивления перегородки может быть заметным даже при малой скорости течения газа. Величина эффекта определяется по разности температур газа $\Delta T$ до и после перегородки.

        \paragraph*{Вывод эффекта Джоуля---Томсона.} Пусть через трубку прошел $\Delta \nu = 1$ mol. газа, пусть $V_1$ и $V_2$ --- молярные объемы газа до и после, $P_1$, $P_2$ --- соответствующие давления. $U_1$, $U_2$ --- внутренние энергии в расчете на 1 mol. Для того чтобы ввести в трубку порцию газа объемом $V_1$ необходимо совершить работу $A_1=P_1V_1$. Выходя из трубки эта же порция совершает работу $A_2=P_2V_2$. Считая, что стенки не проводят тепло и не совершается механической передачи энергии, получаем:
        \begin{equation}
            A_1-A_2=P_1V_1-P_2V_2=(U_2+\mu v_2^2/2)-(U_1+\mu v_1^2/2)
            \label{dA}
        \end{equation}
        здесь кроме изменения внутренней энергии $\Delta U$ учтена кинетическая энергия течения $\mu v_{1,2}^2/2$.

        Определим \emph{молярную энтальпию} газа как $H=U+PV$. Тогда уравнение \eqref{dA} можно переписать:
        \begin{equation}
            H_1-H_2=\frac{\mu}{2}(v_2^2-v_1^2)
            \label{dH}
        \end{equation}
        Это есть уравнение бернулли для течения газа, учитывающее его сжимаемость и внутреннюю энергию.
        
        Внутри перегородки газ испытывает трение. Это приводит к необратимому переходу почти всей кинетической энергии газа в тепловую. Теплообмена с окружающей средой нет, поэтому вся энергия отдается обратно газу и уносится с потоком. Закон сохранения энергии \eqref{dH} остается в силе, однако кинетическая энергия оказывается пренебрежимо малой. Приходим к выводу, что эффект Джоуля---Томсона --- это процесс, в котором энтальпия сохраняется:
        \begin{equation}
            \label{H1=H2}
            H_1\approx H_2
        \end{equation}
        Энтальпия --- функция состояния, зависящая от температуры и от давления. Коэффициентом Джоуля---Томсона называют отношение:
        \begin{equation}
            \label{coefD-T}
            \mu_{\text{J---T}}=\frac{\Delta T}{\Delta P}
        \end{equation}
\end{document}