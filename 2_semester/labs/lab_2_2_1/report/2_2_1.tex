\documentclass[a4paper,12pt]{report}
%general packages
\usepackage[T2A]{fontenc}
\usepackage[utf8]{inputenc}
\usepackage[english,russian]{babel}
\usepackage{circuitikz}
\usepackage{wrapfig}
\usepackage{makecell}
\usepackage{tabularx}
\usepackage{graphicx}
\usepackage{gensymb}
\usepackage{cancel} %cancel symbol
\usepackage{amsmath,amsfonts,amssymb,amsthm,mathtools}
\usepackage[dvipsnames]{xcolor}

%fancy header + geometry
\usepackage{fancyhdr}
\usepackage[a4paper,includehead,nomarginpar,left=15mm,right=15mm,top=15mm,headheight=10mm,bottom=20mm]{geometry}

%pgfplots
\usepackage{pgfplots}
\usepackage{pgfkeys}
\pgfplotsset{compat=1.12}
\usepackage{mathrsfs}

%multi column text
\usepackage{blindtext}
\usepackage{multicol}

%tikz (draw)
\usepackage{tikz}
\usepackage{pstricks-add}
\usetikzlibrary{intersections}
\usetikzlibrary{arrows.meta}
\usetikzlibrary{calc,angles,positioning}
\usetikzlibrary{arrows}
\usepackage{float}

%parskip settings
\parindent=0ex
\setlength{\parskip}{\baselineskip}%
\setlength{\parindent}{0pt}%

%fancy notation for sets
\newcommand{\R}{{\mathbb R}}
\newcommand{\N}{{\mathbb N}}
\newcommand{\fancy}[1]{{\mathbb{#1}}}
%sgn function
\DeclareMathOperator{\sgn}{sgn}

% intersection and union symbols
\newcommand{\uni}{\cup}
\newcommand{\inter}{\cap}
\newcommand{\re}{\text{Re}}
\newcommand{\const}{\text{const}}

\renewcommand{\footrulewidth}{0.4pt}

%\newcommand{\celsius}{$\ ^\circ C$}

%environments

\newtheorem{problem}{Задача}[]
\newenvironment{sol}{\paragraph{Решение}}{}
\renewcommand\thesection{\arabic{section}}

\usepackage{titlesec}
\titlespacing*{\section}
{0cm}{\baselineskip}{0pt}
\titlespacing*{\subsection}
{0pt}{0.1\baselineskip}{0.1\baselineskip}
\titlespacing*{\paragraph}
{0pt}{0.1\baselineskip}{\baselineskip}

\setcounter{secnumdepth}{0}

\begin{document}
	\begin{titlepage}
	\begin{center}
		МОСКОВСКИЙ ФИЗИКО-ТЕХНИЧЕСКИЙ ИНСТИТУТ (НАЦИОНАЛЬНЫЙ ИССЛЕДОВАТЕЛЬСКИЙ УНИВЕРСИТЕТ) \\
		
		
		\hfill \break
		Факультет обшей и прикладной физики\\
		\vspace{2.5cm}
		\LARGE{\textbf{Отчет о выполнении второй лабораторной работе по информатике \\Асимптотическая сложность алогритмов сортировки}}\\
		\hfill \break
		\\
	\end{center}
	
	\begin{flushright}
		Выполнил:\\
		Студент гр. Б02-304\\
		Головинов. Г.А.
	\end{flushright}
	
	\vfill
	
	\begin{center}
		\includegraphics[width=0.15\linewidth]{uni}
	\end{center}
	
	\begin{center} Долгопрудный, 2024 \end{center}
	
	\thispagestyle{empty}
	
\end{titlepage}
	\newpage
	%\pagenumbering{arabic}
    \pagestyle{fancy}

    \fancyhead{}
    \fancyfoot{}
    \fancyhead[L]{\rightmark}
    \fancyhead[R]{\thepage}
    \fancyfoot[R]{Работа 2.5.1. --- Определение коэффициента поверхностного натяжения в зависимости от температуры}

    \section*{Аннотация}
        \paragraph*{Цель работы:} 1) измерить коэффициент поверхностного натяжения в зависимости от температуры; 2) определение полной поверхностной энергии и теплоты, необходимой для изотермического образования единицы поверхности жидкости.
        \paragraph*{В работе используются:} прибор Ребиндера с термостатом; исследуемые жидкости; стаканы; линейка.
    \vspace{0.5cm}
    \hrule
    
    

\end{document}
