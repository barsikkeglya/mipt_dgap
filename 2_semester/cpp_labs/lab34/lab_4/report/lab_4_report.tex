\documentclass[a4paper,12pt]{report}
%general packages
\usepackage[T2A]{fontenc}
\usepackage[utf8]{inputenc}
\usepackage[english,russian]{babel}
\usepackage{circuitikz}
\usepackage{wrapfig}
\usepackage{makecell}
\usepackage{tabularx}
\usepackage{graphicx}
\usepackage{gensymb}
\usepackage{cancel} %cancel symbol
\usepackage{amsmath,amsfonts,amssymb,amsthm,mathtools}
\usepackage[dvipsnames]{xcolor}

%fancy header + geometry
\usepackage{fancyhdr}
\usepackage[a4paper,includehead,nomarginpar,left=15mm,right=15mm,top=15mm,headheight=10mm,bottom=20mm]{geometry}

%pgfplots
\usepackage{pgfplots}
\usepackage{pgfkeys}
\pgfplotsset{compat=1.12}
\usepackage{mathrsfs}

%multi column text
\usepackage{blindtext}
\usepackage{multicol}

%tikz (draw)
\usepackage{tikz}
\usepackage{pstricks-add}
\usetikzlibrary{intersections}
\usetikzlibrary{arrows.meta}
\usetikzlibrary{calc,angles,positioning}
\usetikzlibrary{arrows}
\usepackage{float}

%parskip settings
\parindent=0ex
\setlength{\parskip}{\baselineskip}%
\setlength{\parindent}{0pt}%

%fancy notation for sets
\newcommand{\R}{{\mathbb R}}
\newcommand{\N}{{\mathbb N}}
\newcommand{\fancy}[1]{{\mathbb{#1}}}
%sgn function
\DeclareMathOperator{\sgn}{sgn}

% intersection and union symbols
\newcommand{\uni}{\cup}
\newcommand{\inter}{\cap}

\renewcommand{\footrulewidth}{0.4pt}

%\newcommand{\celsius}{$\ ^\circ C$}

%environments

\newtheorem{problem}{Задача}[]
\newenvironment{sol}{\paragraph{Решение}}{}
\renewcommand\thesection{\arabic{section}}

\usepackage{titlesec}
\titlespacing*{\section}
{0pt}{\baselineskip}{0pt}
\titlespacing*{\paragraph}
{0pt}{\baselineskip}{\baselineskip}
\titlespacing*{\paragraph}
{0pt}{0.1\baselineskip}{\baselineskip}

\setcounter{secnumdepth}{0}

\begin{document}
	\begin{titlepage}
	\begin{center}
		МОСКОВСКИЙ ФИЗИКО-ТЕХНИЧЕСКИЙ ИНСТИТУТ (НАЦИОНАЛЬНЫЙ ИССЛЕДОВАТЕЛЬСКИЙ УНИВЕРСИТЕТ) \\
		
		
		\hfill \break
		Факультет обшей и прикладной физики\\
		\vspace{2.5cm}
		\LARGE{\textbf{Отчет о выполнении второй лабораторной работе по информатике \\Асимптотическая сложность алогритмов сортировки}}\\
		\hfill \break
		\\
	\end{center}
	
	\begin{flushright}
		Выполнил:\\
		Студент гр. Б02-304\\
		Головинов. Г.А.
	\end{flushright}
	
	\vfill
	
	\begin{center}
		\includegraphics[width=0.15\linewidth]{uni}
	\end{center}
	
	\begin{center} Долгопрудный, 2024 \end{center}
	
	\thispagestyle{empty}
	
\end{titlepage}
	\newpage
	%\pagenumbering{arabic}
    \pagestyle{fancy}

    \fancyhead{}
    \fancyfoot{}
    \fancyhead[L]{\rightmark}
    \fancyhead[R]{\thepage}
    \fancyfoot[R]{Лабораторная работа 4 --- структуры данных-2}

    %\begin{multicols}{2}

        \section{Хэш-таблица.}
        \begin{figure}[H]
            \centering
            \begin{tikzpicture}[]
                \begin{axis}[name=plot1,
                    width=0.75\linewidth,
                    title={t(n)},
                    %xmode = log,
                    %ymode = log,
                    xlabel={Size},
                    ylabel={t, ms},
                    legend pos=north west,
                    ymajorgrids=true,
                    xmajorgrids=true,
                    grid style=dashed,]
                \addplot[line width=1pt,solid,color=black] %
                table[x=N,y=iT,col sep=comma]{../data/hash.csv};
                \addlegendentry{Вставка}
                \addplot[line width=1pt,solid,color=red] %
                table[x=N,y=sT,col sep=comma]{../data/hash.csv};
                \addlegendentry{Поиск}
                \addplot[line width=1pt,solid,color=blue] %
                table[x=N,y=dT,col sep=comma]{../data/hash.csv};
                \addlegendentry{Удаление}
                \end{axis}
            \end{tikzpicture}
        \end{figure}


        \section{Несбалансированное дерево.}
        \begin{figure}[H]
            \centering
            \begin{tikzpicture}[]
                \begin{axis}[name=plot1,
                    width=0.75\linewidth,
                    title={t(n)},
                    xmode = log,
                    ymode = log,
                    xlabel={Size},
                    ylabel={t, ms},
                    legend pos=north west,
                    ymajorgrids=true,
                    xmajorgrids=true,
                    grid style=dashed,]
                    \addplot[line width=1pt,solid,color=blue] %
            table[x=n,y=iTa,col sep=comma]{../data/bin_tree.csv};
            \addlegendentry{Вставка}
                \addplot[line width=1pt,solid,color=red] %
                table[x=n,y=sTa,col sep=comma]{../data/bin_tree.csv};
                \addlegendentry{Поиск}
                \addplot[line width=1pt,solid,color=green] %
                table[x=n,y=dTa,col sep=comma]{../data/bin_tree.csv};
                \addlegendentry{Удаление}
                \addplot[line width=1pt,solid,color=pink] %
                table[x=n,y=iTb,col sep=comma]{../data/bin_tree.csv};
                \addlegendentry{Вставка (неравномерно)}
                \addplot[line width=1pt,solid,color=orange] %
                table[x=n,y=sTb,col sep=comma]{../data/bin_tree.csv};
                \addlegendentry{Поиск (неравномерно)}
                \addplot[line width=1pt,solid,color=black] %
                table[x=n,y=dTb,col sep=comma]{../data/bin_tree.csv};
                \addlegendentry{Удаление (неравномерно)}
                \end{axis}
            \end{tikzpicture}
        \end{figure}


        
        \section{AVL дерево.}
        \begin{figure}[H]
            \centering
            \begin{tikzpicture}[]
                \begin{axis}[name=plot1,
                    width=0.75\linewidth,
                    title={t(n)},
                    xmode = log,
                    ymode = log,
                    xlabel={Size},
                    ylabel={t, ms},
                    legend pos=north west,
                    ymajorgrids=true,
                    xmajorgrids=true,
                    grid style=dashed,]
                    \addplot[line width=1pt,solid,color=blue] %
            table[x=N,y=iTa,col sep=comma]{../data/avl.csv};
            \addlegendentry{Вставка}
                \addplot[line width=1pt,solid,color=red] %
                table[x=N,y=sTa,col sep=comma]{../data/avl.csv};
                \addlegendentry{Поиск}
                \addplot[line width=1pt,solid,color=green] %
                table[x=N,y=dTa,col sep=comma]{../data/avl.csv};
                \addlegendentry{Удаление}
                \addplot[line width=1pt,solid,color=pink] %
                table[x=N,y=iTb,col sep=comma]{../data/avl.csv};
                \addlegendentry{Вставка (неравномерно)}
                \addplot[line width=1pt,solid,color=orange] %
                table[x=N,y=sTb,col sep=comma]{../data/avl.csv};
                \addlegendentry{Поиск (неравномерно)}
                \addplot[line width=1pt,solid,color=black] %
                table[x=N,y=dTb,col sep=comma]{../data/avl.csv};
                \addlegendentry{Удаление (неравномерно)}
                \end{axis}
            \end{tikzpicture}
        \end{figure}
       
\end{document}