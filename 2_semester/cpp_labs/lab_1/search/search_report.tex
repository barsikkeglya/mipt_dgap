\documentclass[a4paper,12pt]{article}
\usepackage[T2A]{fontenc}
\usepackage[utf8]{inputenc}
\usepackage[english,russian]{babel}
\usepackage{circuitikz}
\usepackage{wrapfig}
\usepackage{makecell}
\usepackage{tabularx}
\usepackage{graphicx}
\usepackage{gensymb}
\usepackage{cancel} %cancel symbol
\usepackage{amsmath,amsfonts,amssymb,amsthm,mathtools}
\usepackage{pgfplots}
\pgfplotsset{compat=1.15}
\usepackage{mathrsfs}

\setlength{\parskip}{\baselineskip}%
\setlength{\parindent}{0pt}%


%tikz (draw)

\usepackage{tikz}
\usepackage{pstricks-add}
%tikz libraries

\usetikzlibrary{intersections}
\usetikzlibrary{arrows.meta}
\usetikzlibrary{calc,angles,positioning}
\usetikzlibrary{arrows}
\usepackage{float}

\parindent=0ex

\graphicspath{ {C:/Users/George/Documents/MIPT_TEX/} }

\newcommand{\R}{{\mathbb R}}
\newcommand{\N}{{\mathbb N}}
\newcommand{\fancy}[1]{{\mathbb{#1}}}
\DeclareMathOperator{\sgn}{sgn}
\newtheorem{problem}{Задача}[]
\newenvironment{sol}{\paragraph{Решение}}{}
\renewcommand\thesection{\arabic{section}}
\newcommand{\uni}{\cup}
\newcommand{\inter}{\cap}

\begin{document}
	\begin{titlepage}
	\begin{center}
		МОСКОВСКИЙ ФИЗИКО-ТЕХНИЧЕСКИЙ ИНСТИТУТ (НАЦИОНАЛЬНЫЙ ИССЛЕДОВАТЕЛЬСКИЙ УНИВЕРСИТЕТ) \\
		
		
		\hfill \break
		Факультет обшей и прикладной физики\\
		\vspace{2.5cm}
		\LARGE{\textbf{Отчет о выполнении второй лабораторной работе по информатике \\Асимптотическая сложность алогритмов сортировки}}\\
		\hfill \break
		\\
	\end{center}
	
	\begin{flushright}
		Выполнил:\\
		Студент гр. Б02-304\\
		Головинов. Г.А.
	\end{flushright}
	
	\vfill
	
	\begin{center}
		\includegraphics[width=0.15\linewidth]{uni}
	\end{center}
	
	\begin{center} Долгопрудный, 2024 \end{center}
	
	\thispagestyle{empty}
	
\end{titlepage}
	\newpage
	\pagenumbering{arabic}
    
    \section*{Асимптотическая сложность алгоритмов поиска}
    Для выполнения первого задания был написан код (search.cpp), в котором реализованы следующие методы:
    \begin{enumerate}
        \item linear\_search -- линейный поиск
        \item binary\_search -- бинарный поиск
        \item fill -- заполнение массива строго возрастающими целыми положительными числами
        \item generate\_needed -- генерация случайного числа (в массиве или нет, учитывая какой случай нужен)
        \item print\_arr -- выводит весь массив, нужен для debug
        \item timing -- возвращает время, затраченное алгоритмом на выполнение, принимает в качестве аргументов саму функцию поиска, размер тестируемого массива, количество прогонок, флаг среднего или худшего случая.
        \item run\_auto -- делает заданное число полных прогонов (для всех размеров массива), может быть необходимо для уменьшения влияния task handler операционной системы. 
    \end{enumerate}
    Худшая асимптотическая сложность для линейного поиска равна $O(n)$, причем время для среднего случая (если случайные числа распределены равномерно) должно быть в среднем в два раза меньше.
    
    Для бинарного поиска худшая сложность $O(\log n)$.

    
\end{document}