\documentclass[a4paper,12pt]{report}
%general packages
\usepackage[T2A]{fontenc}
\usepackage[utf8]{inputenc}
\usepackage[english,russian]{babel}
\usepackage{circuitikz}
\usepackage{wrapfig}
\usepackage{makecell}
\usepackage{tabularx}
\usepackage{graphicx}
\usepackage{gensymb}
\usepackage{cancel} %cancel symbol
\usepackage{amsmath,amsfonts,amssymb,amsthm,mathtools}
\usepackage[dvipsnames]{xcolor}
\usepackage[normalem]{ulem}
\usepackage{paralist}

%\usepackage{epstopdf} %converting to PDF
%\usepackage{auto-pst-pdf}

%fancy header + geometry
\usepackage{fancyhdr}
\usepackage[a4paper,nomarginpar,left=15mm,right=15mm,top=15mm,headheight=0mm,bottom=15mm]{geometry}

%pgfplots
\usepackage{pgfplots}
\usepackage{pgfkeys}
\pgfplotsset{compat=1.12}
\usepackage{mathrsfs}

%multi column text
\usepackage{blindtext}
\usepackage{multicol}

%tikz (draw)
\usepackage{tikz}
\usepackage{pstricks-add}
\usetikzlibrary{intersections}
\usetikzlibrary{arrows.meta}
\usetikzlibrary{calc,angles,positioning}
\usetikzlibrary{arrows}
\usepackage{float}
\usepackage{filecontents}

%parskip settings
\parindent=0ex
\setlength{\parskip}{\baselineskip}%
\setlength{\parindent}{0pt}%

%fancy notation for sets
\newcommand{\R}{{\mathbb R}}
\newcommand{\N}{{\mathbb N}}
\newcommand{\fancy}[1]{{\mathbb{#1}}}
%sgn function
\DeclareMathOperator{\sgn}{sgn}

% intersection and union symbols
\newcommand{\uni}{\cup}
\newcommand{\inter}{\cap}
\newcommand{\re}{\text{Re}}
\newcommand{\const}{\text{const}}

\renewcommand{\footrulewidth}{0.4pt}

%\newcommand{\celsius}{$\ ^\circ C$}

%environments

\newtheorem{problem}{Задача}[]
\newenvironment{definition}{\paragraph{Определение}}{}
\renewcommand\thesection{\arabic{section}}

\usepackage{titlesec}
\titlespacing*{\section}
{0cm}{\baselineskip}{0pt}
\titlespacing*{\subsection}
{0pt}{0.1\baselineskip}{0.1\baselineskip}
\titlespacing*{\paragraph}
{0pt}{0.1\baselineskip}{\baselineskip}

\setcounter{secnumdepth}{0}

\makeatletter
\newenvironment{sqcases}{%
  \matrix@check\sqcases\env@sqcases
}{%
  \endarray\right.%
}
\def\env@sqcases{%
  \let\@ifnextchar\new@ifnextchar
  \left\lbrack
  \def\arraystretch{1.2}%
  \array{@{}l@{\quad}l@{}}%
}
\makeatother
\begin{document}
	%\begin{titlepage}
	\begin{center}
		МОСКОВСКИЙ ФИЗИКО-ТЕХНИЧЕСКИЙ ИНСТИТУТ (НАЦИОНАЛЬНЫЙ ИССЛЕДОВАТЕЛЬСКИЙ УНИВЕРСИТЕТ) \\
		
		
		\hfill \break
		Факультет обшей и прикладной физики\\
		\vspace{2.5cm}
		\LARGE{\textbf{Отчет о выполнении второй лабораторной работе по информатике \\Асимптотическая сложность алогритмов сортировки}}\\
		\hfill \break
		\\
	\end{center}
	
	\begin{flushright}
		Выполнил:\\
		Студент гр. Б02-304\\
		Головинов. Г.А.
	\end{flushright}
	
	\vfill
	
	\begin{center}
		\includegraphics[width=0.15\linewidth]{uni}
	\end{center}
	
	\begin{center} Долгопрудный, 2024 \end{center}
	
	\thispagestyle{empty}
	
\end{titlepage}
	%\newpage
	%\pagenumbering{arabic}
  %\pagestyle{fancy}

  %\fancyhead{}
  %\fancyfoot{}
  %\fancyhead[L]{}
  %\fancyhead[R]{\thepage}
  %\fancyfoot[R]{Реферат к зачету в рамках Горизонтов физики}

  \section{Что мы прошли в течение семестра?}

  \subsection{Теоретическое введение}
  \begin{compactitem}
      \item Что такое стандартная модель?
      \begin{compactitem}
        \item Основные определения стандартной модели.
        \item История развития стандартной модели.
        \item Предсказывающая сила стандартной модели.
        \item Что стандартная модель не описывает?
        \begin{compactitem}
          \item Темная материя, темная энергия, асимметрия материи-антиматерии и т.д.
        \end{compactitem}
        \item Возможные расширения стандартной модели.
        \begin{compactitem}
          \item Суперсимметрия, стерильные нейтрино и т.д.
        \end{compactitem}
        \item Какова связь эксперимента и теории?
      \end{compactitem} 
      \item Физика нейтрино.
      \begin{compactitem}
        \item Что такое нейтрино, как они были открыты и как они взаимодействуют с другими частицами?
        \item Осцилляция нейтрино.
        \item Поиск стерильного нейтрино.
      \end{compactitem}
      \item Экзотические частицы, тяжелые частицы, прелестные частицы.
      \item Влияние теории на эксперимент.
  \end{compactitem}
  \subsection{Эксперименты}
  \begin{compactitem}
      \item Примеры действующих, и будущих экспериментов, их цель.
      \begin{compactitem}
        \item CMS@LHC, Belle, DANSS, NICA и др.
      \end{compactitem}
      \item Перспективы развития экспериментальной части физики элементарных частиц.
      \begin{compactitem}
        \item Модернизация нынешних экспериментов, планы будущих экспериментов.
      \end{compactitem}
      \item Более <<экзотические>> эксперименты (например, измерение ускорения свободного падения антиводорода).
      \item Как выглядит процесс открытия новых частиц? Как выглядит научный процесс на старших курсах?
      \item Как эксперимент влияет на развитие теории?
  \end{compactitem}
  \subsection{Что мне понравилось больше всего?}
  Короткий ответ: \textbf{Физика космоса}. Одна лекция была посвящена введению в астрофизику (и физику космоса в целом).
  \begin{compactitem}
    \item Какова связь астрофизики и физики элементарных частиц?
    \begin{compactitem}
      \item Темная материя, темная энергия и т.д. --- общие проблемы как для астрофизики, так и для стандартной модели.
    \end{compactitem}
    \item Как астрофизика помогает развитию физики элементарных частиц?
    \begin{compactitem}
      \item Например, космические лучи (сверхбыстрые частицы) позволяют экспериментаторам обнаруживать достаточно тяжелые частицы без необходимости сложных коллайдеров. Так были открыты некоторые частицы стандартной модели.
    \end{compactitem}
    \item Как физика элементарных частиц помогает астрофизике?
    \begin{compactitem}
      \item Например, моделирование взрыва сверхновой невозможно без учета нейтрино, а их масса очень сильно влияет на результат моделирования.
      \item Любое развитие стандартной модели может кардинально изменить существующее представление о большинстве космических феноменов.
    \end{compactitem}
  \end{compactitem}
  
  
\end{document}
