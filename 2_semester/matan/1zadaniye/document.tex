\documentclass[a4paper,12pt]{report}
\usepackage[T2A]{fontenc}
\usepackage[utf8]{inputenc}
\usepackage[english,russian]{babel}
\usepackage{circuitikz}
\usepackage{wrapfig}
\usepackage{makecell}
\usepackage{tabularx}
\usepackage{graphicx}
\usepackage{gensymb}
\usepackage{cancel} %cancel symbol
\usepackage{amsmath,amsfonts,amssymb,amsthm,mathtools}
\usepackage{pgfplots}
\pgfplotsset{compat=1.15}
\usepackage{mathrsfs}

%tikz (draw)

\usepackage{tikz}
\usepackage{pstricks-add}
%tikz libraries

\usetikzlibrary{intersections}
\usetikzlibrary{arrows.meta}
\usetikzlibrary{calc,angles,positioning}
\usetikzlibrary{arrows}
\usepackage{float}

\parindent=0ex

\graphicspath{ {C:/Users/George/Documents/MIPT_TEX/} }

\newcommand{\R}{{\mathbb R}}
\newcommand{\N}{{\mathbb N}}
\newcommand{\fancy}[1]{{\mathbb{#1}}}
\DeclareMathOperator{\sgn}{sgn}
\newtheorem{problem}{Задача}[]
\newenvironment{sol}{\paragraph{Решение}}{}
\renewcommand\thesection{\arabic{section}}
\newcommand{\uni}{\cup}
\newcommand{\inter}{\cap}

\begin{document}
	\begin{titlepage}
	\begin{center}
		МОСКОВСКИЙ ФИЗИКО-ТЕХНИЧЕСКИЙ ИНСТИТУТ (НАЦИОНАЛЬНЫЙ ИССЛЕДОВАТЕЛЬСКИЙ УНИВЕРСИТЕТ) \\
		
		
		\hfill \break
		Факультет обшей и прикладной физики\\
		\vspace{2.5cm}
		\LARGE{\textbf{Отчет о выполнении второй лабораторной работе по информатике \\Асимптотическая сложность алогритмов сортировки}}\\
		\hfill \break
		\\
	\end{center}
	
	\begin{flushright}
		Выполнил:\\
		Студент гр. Б02-304\\
		Головинов. Г.А.
	\end{flushright}
	
	\vfill
	
	\begin{center}
		\includegraphics[width=0.15\linewidth]{uni}
	\end{center}
	
	\begin{center} Долгопрудный, 2024 \end{center}
	
	\thispagestyle{empty}
	
\end{titlepage}
	\newpage
	\pagenumbering{arabic}
    
    \begin{problem}[К3.2.9(3)]
        Является ли множество, на котором определена функция $u(x,y)$
        а) замкнутым, б) открытым, в) линейно связным,
        г) областью, д) замкнутой областью, е) выпуклым?
        \[
            u(x,y)=\ln (1-2x-x^2-y^2)  
        \]
    \end{problem}
    \begin{sol}
        Область определения логарифма:
        \begin{equation*}
            1-2x-x^2-y^2>0\Leftrightarrow (x+1)^2+y^2<2
        \end{equation*}
        Открытый шар-2 радуиса $\sqrt{2}$ с центром (-1,0)
        \begin{figure}[H]
            \centering
            \definecolor{qqqqff}{rgb}{0.,0.,1.}
            \begin{tikzpicture}[line cap=round,line join=round,>=triangle 45,x=1.0cm,y=1.0cm]
            \begin{axis}[
            x=1.0cm,y=1.0cm,
            axis lines=middle,
            ymajorgrids=true,
            xmajorgrids=true,
            xmin=-4.0,
            xmax=4.0,
            ymin=-4.0,
            ymax=4.0,
            xtick={-4.0,-3.0,...,4.0},
            ytick={-4.0,-3.0,...,4.0},]
            \clip(-4.,-4.) rectangle (4.,4.);
            \draw [rotate around={0.:(-1.,0.)},line width=1.pt,dash pattern=on 5pt off 5pt,color=qqqqff,fill=qqqqff,fill opacity=0.25] (-1.,0.) ellipse (1.4142135623730951cm and 1.4142135623730951cm);
            \end{axis}
            \end{tikzpicture}
        \end{figure}
        Множество не является замкнутым, так как оно не совпадает со своим замыканием. Открыто, так как шар открытый. Линейно связно, является областью, но не замкнутой областью. Выпукло.
    \end{sol}

    \begin{problem}[К3.2.39]
        \begin{center}
            \includegraphics*[width=1\linewidth]{img/2.39.png}
        \end{center}
    \end{problem}
    \begin{sol}
        
    \end{sol}

    
    \begin{problem}[Т14]
        Найдите вторые частные производные функции в данной точке (0,0,0)
        \begin{equation*}
            f(x,y,z)=(1+x)^{\alpha}(1+y)^{\beta}(1+z^\gamma)
        \end{equation*}
    \end{problem}
    \begin{sol}
        \begin{equation*}
            \ln f =\alpha \ln(1+x) + \beta \ln(1+y)+\ln(1+z^\gamma)
        \end{equation*}
        \begin{equation*}
            d\ln f=\frac{\alpha dx}{1+x}+\frac{\beta dy}{1+y}+\frac{\gamma z^{\gamma-1dz}}{1+z^\gamma}
        \end{equation*}
        \begin{equation*}
            d\ln f=\frac{df}{f}\rightarrow df(0,0,0)=f(0,0,0)(\alpha dx+\beta dy)
        \end{equation*}
        \begin{equation}
            df(0,0,0)=\alpha dx+\beta dy
        \end{equation}
        \begin{equation*}
            d^2f=d(df) \quad d^2f=d(fd\ln f)
        \end{equation*}
        \begin{equation*}
            d^2f(x,y,z)=df(x,y,z)\otimes\frac{df(x,y,z)}{f(x,y,z)}+f(x,y,z)d^2\ln f(x,y,z)
        \end{equation*}
        \begin{equation*}
            \frac{df(0,0,0)\otimes df(0,0,0)}{f(0,0,0)}=(\alpha dx+\beta dy)^2=\alpha^2dx\otimes dx+2\alpha \beta dx\otimes dy+\beta^2dy\otimes dy
        \end{equation*}
        \begin{equation*}
            fd^2\ln f(0,0,0)=1\cdot d\left(\frac{\alpha dx}{1+x}+\frac{\beta dy}{1+y}+\frac{\gamma z^{\gamma-1}dz}{1+z^\gamma} \right)
        \end{equation*}
        \begin{equation*}
            fd^2\ln f(0,0,0)=-\frac{\alpha dx^2}{(1+x)^2}-\frac{\beta dy^2}{(1+y)^2}+\frac{\gamma (\gamma-1)z^{\gamma-1}[z+z^{\gamma+1}-z\cdot\gamma z^{\gamma-1}]dz^2}{(1+z^\gamma)^2}
        \end{equation*}
        Итого
        \begin{equation*}
            d^2f(0,0,0)=\alpha^2dx^2+2\alpha\beta dxdy+\beta^2dy^2-\alpha dx^2 -\beta dy^2
        \end{equation*}
    \end{sol}
	
    \begin{problem}[Т15а]
        Найдите вторые частные производные в точке (1,1) функции $f(x,y)$ заданной неявно соотношением
        $$ ef=e^{x+y+f} $$
    \end{problem}
    \begin{sol}
        \begin{eqnarray*}
            edf=&e^{x+y+f}(dx+dy+df)\\ 
            edf-efdf=&ef(dx+dy)\\ 
            df=&(1-f)^{-1}f(dx+dy)\\ 
            d^2f=&(1-f)^{-2}(dfdx+dfdy+fdx^2+fdy^2)(1-f)+(f(dx+dy)df)
        \end{eqnarray*}
        \begin{eqnarray*}
            d^2f=(1-f)^{-2}((fdx^2+fdxdy+fdx^2+fdy^2)+\\ 
            +(1-f)^{-1}(f^2dx^2+2f^2dxdy+f^2dy^2))
        \end{eqnarray*}
        Найдем $f(1,1)$:
        \begin{eqnarray*}
            f&=&e^{1+f}\\ 
            \ln f&=&1+f\ (??)
        \end{eqnarray*}
    \end{sol}
    \begin{problem}[Т15б]
        Найдите вторые частные производные в точке $(1,1)$ функции $f(x,y)$ заданной неявно соотношением
        \[
          f^3-3xyf-2=0  
        \]
    \end{problem}
    \begin{sol}
        \begin{equation*}
            3f^2df-3yfdx-3xfdy-3xydf=0
        \end{equation*}
        \begin{equation*}
            df(3f^2-3xy)=3yfdx+3xfdy
        \end{equation*}
        \begin{equation*}
            df=\frac{yfdx+xfdy}{f^2-xy}=\frac{2}{3}(dx+dy)
        \end{equation*}
        \begin{eqnarray*}
            d^2f=\frac{1}{(f^2-xy)^2}((fdxdy+ydxdf+fdxdy+xdydf)(f^2-xy)-
            \\-(yfdx+xfdy)(2fdf-xdy-ydx))
        \end{eqnarray*}
        \begin{equation*}
            d^2f(1,1)=\frac{1}{9}(3(4dxdy+df(dx+dy))-2(dx+dy)(4df-dx-dy))
        \end{equation*}
        \begin{equation*}
            d^2f(1,1)=\frac{12dxdy-\frac{4}{3}(dx+dy)^2}{9}=\frac{-4}{27}dx^2-\frac{4}{27}dy^2+\frac{28}{27}dxdy
        \end{equation*}
        Тогда
        \begin{eqnarray}
            \frac{\partial^2f}{\partial x^2}=\frac{-4}{27}\\ 
            \frac{\partial^2f}{\partial y^2}=\frac{-4}{27}\\ 
            \frac{\partial^2f}{\partial x\partial y}=\frac{14}{27}
        \end{eqnarray}
    \end{sol}


    \begin{problem}[Т16]
        Найдите частные производные всех порядков функции $f(x,y,z)=\ln{(x+y+z)}$
    \end{problem}
    \begin{sol}
        Положим $t=x+y+z$, тогда $f(t)=\ln(t)$
        \begin{equation}
            d^nf=\frac{(-1)^{n-1}(n-1)!}{t^n}dt^n
        \end{equation}
        В свою очередь 
        \begin{equation}
            dt^n=(dx+dy+dz)^n=\displaystyle\sum_{k_1,k_2,k_3}^{}{n\choose k_1,k_2,k_3}dx^{k_1}dy^{k_2}dz^{k_3}
        \end{equation}
        Чтобы найти некоторую частную производную необходимо выбрать $k_1,k_2,k_3$, такие что $k_1+k_2+k_3=n$, и она будет равна
        \begin{equation}
            \frac{\partial^n f}{\partial x^{k_1}\partial y^{k_2}\partial z^{k_3}}=\frac{(-1)^{n-1}(n-1)!}{(x+y+z)^n}
        \end{equation}
    \end{sol}
    
\end{document}