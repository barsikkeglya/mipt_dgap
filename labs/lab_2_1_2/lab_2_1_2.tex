\documentclass[a4paper,12pt]{article}
\usepackage[T2A]{fontenc}
\usepackage[utf8]{inputenc}
\usepackage[english,russian]{babel}
\usepackage{circuitikz}
\usepackage{wrapfig}
\usepackage{makecell}
\usepackage{tabularx}
\usepackage{graphicx}
\usepackage{gensymb}
\usepackage{cancel} %cancel symbol
\usepackage{amsmath,amsfonts,amssymb,amsthm,mathtools}

\setlength{\parskip}{\baselineskip}%
\setlength{\parindent}{0pt}%

%tikz (draw)

\usepackage{tikz}

%tikz libraries

\usetikzlibrary{intersections}
\usetikzlibrary{arrows.meta}
\usetikzlibrary{calc,angles,positioning}

\usepackage{float}

\parindent=0ex

%\graphicspath{ {C:/Users/George/Documents/MIPT_TEX/} }

\newcommand{\R}{{\mathbb R}}
\newcommand{\N}{{\mathbb N}}
\newcommand{\fancy}[1]{{\mathbb{#1}}}
\DeclareMathOperator{\sgn}{sgn}
\newtheorem{problem}{Задача}[]
\newenvironment{sol}{\paragraph{Решение}}{}
\renewcommand\thesection{\arabic{section}}
\newcommand{\uni}{\cup}
\newcommand{\inter}{\cap}

\begin{document}
	\begin{titlepage}
	\begin{center}
		МОСКОВСКИЙ ФИЗИКО-ТЕХНИЧЕСКИЙ ИНСТИТУТ (НАЦИОНАЛЬНЫЙ ИССЛЕДОВАТЕЛЬСКИЙ УНИВЕРСИТЕТ) \\
		
		
		\hfill \break
		Факультет обшей и прикладной физики\\
		\vspace{2.5cm}
		\LARGE{\textbf{Отчет о выполнении второй лабораторной работе по информатике \\Асимптотическая сложность алогритмов сортировки}}\\
		\hfill \break
		\\
	\end{center}
	
	\begin{flushright}
		Выполнил:\\
		Студент гр. Б02-304\\
		Головинов. Г.А.
	\end{flushright}
	
	\vfill
	
	\begin{center}
		\includegraphics[width=0.15\linewidth]{uni}
	\end{center}
	
	\begin{center} Долгопрудный, 2024 \end{center}
	
	\thispagestyle{empty}
	
\end{titlepage}
	\newpage
	\pagenumbering{arabic}
    
    \subsection*{Аннотация}
        \paragraph*{Цель работы:} определение отношения $C_p/C_v$ для воздуха или углекислого газа по измерению давления в стеклянном сосуде. Измерения производятся сначала для адиабатического расширения гааз, а затем после нагревания сосуда и газа до комнатной температуры.
        \paragraph*{В работе используются:} стеклянный сосуд; U-образный жидкостный манометр; резиновая группа; газгольдер с углексилым газом.
    \section{Основные теоретические сведения}
    \subsection*{Экспериментальная установка}
        \begin{figure}[H]
            \includegraphics*[width=0.75\textwidth]{ustanovka.png}
            \centering
            \caption{Установка для определения отношения $C_p/C_v$}
        \end{figure}
        На рисунке сосуд А (объем $\approx$ 20 л.), кран К, U-образный жидкостный манометр М. Кран К$_1$ и резиновая груша позволяют создавать избыточное давление воздуха. Углекислый газ подается из газгольдера.

        В начале опыта газ в сосуде А находится при комнатной температуре $T_1$, давлении $P_1$, несколько превышающем атмосферное давление $P_0$. После открытия крана К давление и температура газа будут понижаться.

        Этот процесс приближенно можно считать адиабатическим. Приближение основано на том, что равновесие в газах по давлению наступает намного быстрее, чем равновесие по температуре. Соответсвтенно будем считать $\Delta t_P$ -- время установления равновесия по давлению сильно меньше, чем $\Delta t_T$ -- время установления равновесия по температуре.

        

\end{document}