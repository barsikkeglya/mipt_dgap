\documentclass[a4paper,12pt]{report}
\usepackage[T2A]{fontenc}
\usepackage[utf8]{inputenc}
\usepackage[english,russian]{babel}
\usepackage{circuitikz}
\usepackage{wrapfig}
\usepackage{makecell}
\usepackage{tabularx}
\usepackage{graphicx}
\usepackage{gensymb}
\usepackage{cancel} %cancel symbol
\usepackage{amsmath,amsfonts,amssymb,amsthm,mathtools}

%tikz (draw)

\usepackage{tikz}

%tikz libraries

\usetikzlibrary{intersections}
\usetikzlibrary{arrows.meta}
\usetikzlibrary{calc,angles,positioning}

\usepackage{float}

\parindent=0ex

\graphicspath{ {C:/Users/George/Documents/MIPT_TEX/} }

\newcommand{\R}{{\mathbb R}}
\newcommand{\N}{{\mathbb N}}
\newcommand{\fancy}[1]{{\mathbb{#1}}}
\DeclareMathOperator{\sgn}{sgn}
\newtheorem{problem}{Задача}[]
\newenvironment{sol}{\paragraph{Решение}}{}
\renewcommand\thesection{\arabic{section}}
\newcommand{\uni}{\cup}
\newcommand{\inter}{\cap}

\begin{document}
	\begin{titlepage}
	\begin{center}
		МОСКОВСКИЙ ФИЗИКО-ТЕХНИЧЕСКИЙ ИНСТИТУТ (НАЦИОНАЛЬНЫЙ ИССЛЕДОВАТЕЛЬСКИЙ УНИВЕРСИТЕТ) \\
		
		
		\hfill \break
		Факультет обшей и прикладной физики\\
		\vspace{2.5cm}
		\LARGE{\textbf{Отчет о выполнении второй лабораторной работе по информатике \\Асимптотическая сложность алогритмов сортировки}}\\
		\hfill \break
		\\
	\end{center}
	
	\begin{flushright}
		Выполнил:\\
		Студент гр. Б02-304\\
		Головинов. Г.А.
	\end{flushright}
	
	\vfill
	
	\begin{center}
		\includegraphics[width=0.15\linewidth]{uni}
	\end{center}
	
	\begin{center} Долгопрудный, 2024 \end{center}
	
	\thispagestyle{empty}
	
\end{titlepage}
	\newpage
	\pagenumbering{arabic}
    
    \begin{problem}[Т14]
        Найдите вторые частные производные функции в данной точке
        \begin{equation*}
            f(x,y,z)=(1+x)^{\alpha}(1+y)^{\beta}(1+z)^{\gamma}
        \end{equation*}
    \end{problem}
    \begin{sol}
        \begin{eqnarray*}
            df&=&\alpha(1+x)^{\alpha-1}(1+y)^\beta (1+z)^\gamma dx+\\ 
            &+&\beta(1+y)^{\beta-1}(1+x)^\alpha(1+z)^\gamma dy+\\ 
            &+&\gamma(1+z)^{\gamma-1}(1+x)^\alpha(1+y)^\beta dz
        \end{eqnarray*}
        \begin{eqnarray*}
            d^2f&=\alpha(\alpha-1)(1+x)^{\alpha-2}(1+y)^\beta(1+z)^\gamma dx\otimes dx&+\\ 
            &+\alpha\beta(1+x)^{\alpha-1}(1+y)^{\beta-1}(1+z)^\gamma dx\otimes dy&+\\
            &+\alpha\gamma(1+x)^{\alpha-1}(1+y)^\beta(1+z)^{\gamma-1}dx\otimes dz&+\\ 
            &+\beta(\beta-1)(1+x)^\alpha(1+y)^{\beta-2}(1+z)^\gamma dy\otimes dy&+\\ 
            &+\alpha\beta(1+x)^{\alpha-1}(1+y)^{\beta-1}(1+z)^\gamma dx\otimes dy &+\\ 
            &+\beta\gamma(1+x)^\alpha (1+y)^{\beta-1}(1+z)^{\gamma-1}dy\otimes dz&+\\ 
            &+\gamma(\gamma-1)(1+x)^\alpha(1+y)^\beta(1+z)^{\gamma-2} dz\otimes dz&+\\ 
            &+\alpha\gamma(1+x)^{\alpha-1}(1+y)^\beta(1+z)^{\gamma-1}dx\otimes dz&+\\ 
            &+\beta\gamma(1+x)^\alpha(1+y)^{\beta-1}(1+z)^{\gamma-1}dy\otimes dz&+
        \end{eqnarray*}
    \end{sol}
	
    \begin{problem}[Т15а]
        Найдите вторые частные производные в точке (1,1) функции $f(x,y)$ заданной неявно соотношением
        $$ ef=e^{x+y+f} $$
    \end{problem}
    \begin{sol}
        \begin{eqnarray*}
            edf=&e^{x+y+f}(dx+dy+df)\\ 
            edf-efdf=&ef(dx+dy)\\ 
            df=&(1-f)^{-1}f(dx+dy)\\ 
            d^2f=&(1-f)^{-2}(dfdx+dfdy+fdx^2+fdy^2)(1-f)+(f(dx+dy)df)
        \end{eqnarray*}
        \begin{eqnarray*}
            d^2f=(1-f)^{-2}((fdx^2+fdxdy+fdx^2+fdy^2)+\\ 
            +(1-f)^{-1}(f^2dx^2+2f^2dxdy+f^2dy^2))
        \end{eqnarray*}
        Найдем $f(1,1)$:
        \begin{eqnarray*}
            f&=&e^{1+f}\\ 
            \ln f&=&1+f\ (??)
        \end{eqnarray*}
    \end{sol}
    \begin{problem}[Т15б]
        Найдите вторые частные производные в точке $(1,1)$ функции $f(x,y)$ заданной неявно соотношением
        \[
          f^3-3xyf-2=0  
        \]
    \end{problem}
    \begin{sol}
        \begin{equation*}
            3f^2df-3yfdx-3xfdy-3xydf=0
        \end{equation*}
        \begin{equation*}
            df(3f^2-3xy)=3yfdx+3xfdy
        \end{equation*}
        \begin{equation*}
            df=\frac{3yfdx+3xfdy}{3f^2-3xy}=\frac{2}{3}(dx+dy)
        \end{equation*}
        \begin{eqnarray*}
            d^2f=\frac{(6fdxdy+3ydxdf+3xdydf)(3f^2-3xy)}{(3f^2-3xy)^2}-\\-\frac{(3yfdx+3xfdy)(6fdf-3ydx-3xdy)}{(3f^2-3xy)^2}\\ 
            d^2f=\frac{(12dxdy+2dx(dx+dy)+2dy(dx+dy))\cdot9}{9^2}-\\ 
            -\frac{6(dx+dy){5(dx+dy)}}{9^2}\\ 
            d^2f=\frac{144dxdy+18dx^2+18dy^2}{9^2}-\frac{30dx^2+60dxdy+30dy^2}{9^2}\\ 
            d^2f=\frac{1}{81}(84dxdy-12dx^2-12dy^2)
        \end{eqnarray*}
        Тогда
        \begin{eqnarray}
            \frac{\partial^2f}{\partial x^2}=\frac{-12}{81}\\ 
            \frac{\partial^2f}{\partial y^2}=\frac{-12}{81}\\ 
            \frac{\partial^2f}{\partial x\partial y}=\frac{42}{81}
        \end{eqnarray}
    \end{sol}


    \begin{problem}[Т16]
        Найдите частные производные всех порядков функции $f(x,y,z)=\ln{(x+y+z)}$
    \end{problem}
    \begin{sol}
        Положим $t=x+y+z$, тогда $f(t)=\ln(t)$
        \begin{equation}
            d^nf=\frac{(-1)^{n-1}(n-1)!}{t^n}dt^n
        \end{equation}
        В свою очередь 
        \begin{equation}
            dt^n=(dx+dy+dz)^n=\displaystyle\sum_{k_1,k_2,k_3}^{}{n\choose k_1,k_2,k_3}dx^{k_1}dy^{k_2}dz^{k_3}
        \end{equation}
        Чтобы найти некоторую частную производную необходимо выбрать $k_1,k_2,k_3$, такие что $k_1+k_2+k_3=n$, и она будет равна
        \begin{equation}
            \frac{\partial^n f}{\partial x^{k_1}\partial y^{k_2}\partial z^{k_3}}=\frac{(-1)^{n-1}(n-1)!}{(x+y+z)^n}
        \end{equation}
    \end{sol}
    
\end{document}